\documentclass[hyperref={pdfpagelabels=false},11pt]{beamer}
\let\Tiny=\tiny
\usetheme{PaloAlto} %Goettingen
\usecolortheme{seahorse}
\usefonttheme{professionalfonts}
\usepackage[utf8]{inputenc}
\usepackage[brazil]{babel}
\usepackage{ucs}
\usepackage{amsmath}
\usepackage{amsfonts}
\usepackage{amssymb}
\author{Thales Ramon, Ártus Bolzanni, Vitor Pinho}
\title{Lavoura Car}
\subtitle{Relatório do Sprint 2, 3 e  4}
\date[UNEB-SI 2010.2]{Segunda Apresentação da Matéria Tópicos Especiais em Engenharia de Software}
\institute[UNEB]{Universidade Estadual da Bahia}
\newtheorem{confirmations}[theorem]{Confirmations}
\begin{document}
% Nosso foco será uma apresentação de 40 minutos. Temos que pensar bem em como dividir os slides.
% Vamos tentar fazer os slides terem conteúdo de no máximo 3 a 4 minutos, embora saibamos que com diagramas vai dar bem mais.
	\begin{frame}
		\titlepage
	\end{frame}
	
%A apresentação será dividida em 4 partes.	
	\part{Sprint 2}
	%Essa parte é mais uma revisão. Pense narrador de dragonball z.
		\begin{frame}
			\frametitle{Tópicos do Sprint 2}
			\tableofcontents[pausesections]
		\end{frame}
		
		\section{Sprint Backlog 2}
	
		%A não ser por essa parte. Não falamos dela, não custa nada apresentar a tecnologia. Mas deixa javascript para depois.
			\begin{frame}
				\frametitle{User stories -- Trajetos}
				\begin{block}{Item 1}<1->
					Como usuário eu quero ligar um ou mais trajetos a um dia específico
					
				\begin{confirmations}
						{Usuário seleciona o dia e os trajetos que devem ser associados.}
				\end{confirmations}	

				\end{block}
				\begin{block}{Item 4}<2->
					Como usuário eu quero gravar os meus trajetos
					
					\begin{confirmations}
						\begin{itemize}
							\item Usuário informa coordenadas no mapa e o nome do trajeto.
							\item Um conjunto de coordenadas são salvas como um trajeto.
						\end{itemize}
					\end{confirmations}
				\end{block} 								
			\end{frame}
			
			\begin{frame}
				\frametitle{User Stories -- Autenticação}	
				\begin{block}{Item 15}
					Como usuário, quero poder me autenticar no site via e-mail e senha.
					
					\begin{confirmations}
						\begin{itemize}
							\item Usuário informa e-mail e senha válidos.
							\item Sistema retorna erro se não for válido.
						\end{itemize}
					\end{confirmations}
				\end{block}
			\end{frame}			
			
			
				
		\section{O Google Maps Api}
			
			\begin{frame}
				\frametitle{O Que é o Google Maps API}
				\begin{block}{Definição}
					\begin{itemize}
						\item API para dar aos desenvolvedores várias formas de embutir e personalizar o google maps em páginas
						\item Feito em JavaScript
						\item Suporte em vários browser e em celular
					\end{itemize}	
				\end{block}						
			\end{frame}
			
		\section{Diagramas da fase de projeto}
			\subsection{Diagrama de Classes}
			\subsection{Diagrama de Entidade-Relacionamento}
			\subsection{Diagrama de Atividades}
	
	\part{Sprint 3}
	
		\begin{frame}
			\frametitle{Tópicos do Sprint 3}
			\tableofcontents[pausesections]
		\end{frame}	
	
		\section{Sprint Backlog 3}
			\begin{frame}{User Stories -- Carros}
				\begin{block}{Item 5}<1->
					Como usuário, quero poder cadastrar o meu automóvel.
					\begin{confirmations}
						\begin{itemize}
							\tiny\item Usuário seleciona as especificações de seu carro e as peças componentes
							\tiny\item O sistema grava e associa o carro a todas as suas informações.
						\end{itemize}
					\end{confirmations}
				\end{block}
				
				\begin{block}{Item 7}<2->
					Como admin, quero poder cadastrar modelos de carros e suas respectivas peças
					\begin{confirmations}
						\begin{itemize}
							\tiny\item O admin fornecerá dados de um determinado carro ou peça
							\tiny\item O sistema disponibilizará os carros e peças para o usuário
						\end{itemize}
					\end{confirmations}
				\end{block}
			\end{frame}
			
			\begin{frame}{User Stories -- Dias}
				\begin{block}{Item 1}
					Como usuário, quero poder associar os meus trajetos aos dias que os farei.
					\begin{confirmations}
						\begin{itemize}
							\item O usuário seleciona o carro para associar um trajeto e informa quais dias pretende executar o trajeto.
							\item O sistema faz a associação e confirma para o usuário
						\end{itemize}
					\end{confirmations}
				\end{block}
			\end{frame}
					
		% Aqui que vamos falar de javascript. E ajax. Tocar no assunto, não dar uma aula sobre isso :B Faculdade não é lugar pra isso ;D
		\section{Javascript e outros}
			\subsection{Javascript}
			\subsection{Ajax}
			\subsection{JSON}
		\section{Diagramas da fase de projeto}
		
	\part{Sprint 4}
		
		\begin{frame}
			\frametitle{Tópicos do Sprint 4}
			\tableofcontents[pausesections]
		\end{frame}
			
		%Essa parte tem que ser finalizada depois. Quando estiver pronto o acabamento, que são "tarefas" do sprint 4.
 		\section{Sprint Backlog 4}
			\begin{frame}{User Stories -- Trocas}
				\begin{block}{Item 6}
					Como usuário, quero informar que troquei determinada peça de meu carro
					\begin{confirmations}
						\begin{itemize}
							\item O usuário marca quais peças trocar e seleciona a nova peça
							\item O sistema atualiza o carro para utilizar essa nova peça
						\end{itemize}
					\end{confirmations}
				\end{block}
			\end{frame}
			
			\begin{frame}{User Stories -- Cálculo}
				\begin{block}{Item 2}
					Como usuário, eu quero que o sistema compute os trajetos já realizados, atualizando as condições de minhas peças e o a quilometragem do meu carro.
					\begin{confirmations}
						Sistema calcula automaticamente e atua os carros e peças em conformidade com os trajetos já feitos
					\end{confirmations}
				\end{block}
			\end{frame}
			
			
				 		
 		
 		%Finalmente um lugar para eu estrear o pacote math do latex :B
		\section{O cálculo}
		\section{Diagramas da fase de projeto}
	\part{Sprint 5}
		\section{Sprint Backlog 5}
			\begin{frame}
				\frametitle{User Stories -- Histórico}
				\begin{block}{Item 12}
					Como usuário, quero saber quantas e quais trocas eu já fiz em um determinado carro.
					\begin{confirmations}
						\begin{itemize}
							\item{Usuário seleciona um determinado carro para exibir o histórico}
							\item{Sistema retorna dados de todas as trocas já efetuadas}
						\end{itemize}
					\end{confirmations}
				\end{block}
			\end{frame}		
		
		\section{Interface}
		\section{Diagramas da fase de projeto}
	\part{A aplicação}
		%Porque se falarmos da aplicação bastante como se fosse um produto comercial, ela acaba se tornando um. Peter Pan me ensinou isso.
		\section{Proposta e entrega}
			\subsection{O backlog do produto}
			\subsection{O protótipo de interface}
			\subsection{O diagrama conceitual}
		\section{Direções futuras}
			\subsection{O deploy}
			\subsection{Melhoramentos}
		\section{Conclusão}
			\begin{center}
				\textbf{Aberto a questionamentos}
			\end{center}
		%E aqui embaixo adicionamos bibliografia :B
\end{document}